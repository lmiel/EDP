\documentclass{article}
\usepackage{lipsum}
\usepackage[spanish]{babel} % Paquete para configurar el idioma en español
\usepackage{titletoc} % Paquete para personalizar la generación de la tabla de contenido
\usepackage[a4paper, margin=4cm]{geometry} % Paquete para ajustar los márgenes de página

\setlength{\parskip}{12pt} % Ajusta el valor según tus preferencias


\title {\huge \textbf{El teorema espectral en EDP lineales}}
\author{Lucía Mielgo}
\date{\today}
\begin{document}
\maketitle % Genera la portada
\newpage

\renewcommand{\contentsname}{Índice} 
\tableofcontents
\newpage
\section{Introducción}
Este trabajo constituye una breve introducción al Teorema espectral aplicado a las ecuaciones en derivadas parciales lineares.

Las ecuaciones en derivadas parciales (EDP) se presentan comúnmente como modelos de evolución en la ciencia y la tecnología. Estos modelos describen cómo cambia una determinada cantidad o variable a lo largo del tiempo. Esta cantidad o variable puede representar una amplia variedad de objetos y fenómenos, desde la ubicación de un satélite en el espacio hasta el comportamiento de un átomo, pasando por los índices financieros o el impacto de una enfermedad en una población. En pocas palabras, los modelos de evolución capturan la idea fundamental de cómo los sistemas se transforman y se desarrollan a lo largo del tiempo, reflejando nuestra comprensión de su dinámica.

El teorema espectral en su forma general para operadores autoadjuntos fue establecido por David Hilbert en la primera década del siglo XX. Hilbert demostró que todo operador autoadjunto en un espacio de Hilbert tiene una base ortogonal de autovectores asociados a autovalores reales.

El teorema espectral, también conocido como teorema de diagonalización, es un resultado esencial en el álgebra lineal y el análisis funcional. Establece que ciertos operadores lineales autoadjuntos en espacios de Hilbert tienen una base ortogonal de autovectores asociados con autovalores correspondientes. En el contexto de las EDP, el teorema espectral proporciona información importante sobre las soluciones y propiedades de estas ecuaciones.

El objetivo principal de este trabajo de profundización es explorar y analizar la aplicación del teorema espectral en el estudio de las ecuaciones en derivadas parciales. Se pretende investigar cómo el teorema espectral puede utilizarse para comprender mejor las propiedades de las soluciones de EDP, así como para abordar problemas de autovalores asociados a operadores lineales involucrados en las ecuaciones.

\newpage

\section{Fundamentos teóricos}
    \subsection{Breve introducción a las EDP y su importancia de enteros y multiplicación clásica}
    Las ecuaciones parabólicas y hiperbólicas, como la ecuación del calor y la ecuación de ondas, respectivamente, son los modelos más clásicos y representativos en el contexto de las ecuaciones en derivadas parciales de evolución. Estas ecuaciones tienen características matemáticas muy diferentes. Mientras que la ecuación del calor describe fenómenos altamente irreversibles en el tiempo, donde la información se propaga a una velocidad infinita, la ecuación de ondas es el prototipo de un modelo de propagación con velocidad finita y completamente reversible en el tiempo. En resumen, la ecuación del calor captura procesos de difusión y disipación, mientras que la ecuación de ondas representa la propagación de perturbaciones a través de un medio.
    \subsection{Conceptos básicos del teorema espectral y su aplicación en problemas de autovalores}

\newpage

\section{El teorema espectral en EDP lineales}
    \subsection{Análisis de las EDP lineales y su formulación matemática}
    \subsection{Descripción del teorema espectral en el contexto de EDP lineales}
    \subsection{Ejemplos y aplicaciones del teorema espectral en EDP lineales}

\newpage

\section{Conclusiones}
\newpage

\section{Referencias bibliográficas}

\newpage




\end{document}

