\documentclass{article}

\title{Fourier aplicado al algoritmo de Schönhage-Strassen}
\author{Lucía Mielgo}
\date{\today}
\begin{document}

\maketitle

\newpage
\tableofcontents

\section{Introducción}
    \subsection{Contexto y motivación}
    \subsection{Objetivos del trabajo}
    
\section{Fundamentos teóricos}
    \subsection{Aritmética de enteros y multiplicación clásica}
    \subsection{Transformada rápida de Fourier (FFT)}
    \subsection{Aritmética modular}
    
\section{Descripción del algoritmo de Schönhage-Strassen}
    \subsection{Explicación general del enfoque y estrategia}
    \subsection{Descomposición en polinomios}
    \subsection{Multiplicación de polinomios mediante FFT}
    \subsection{Manejo de la aritmética modular}
    
\section{Análisis de complejidad}
    \subsection{Complejidad asintótica del algoritmo}
    \subsection{Comparación con la multiplicación clásica}
    \subsection{Estudio de las constantes ocultas y su impacto práctico}
    
\section{Implementación y consideraciones prácticas}
    \subsection{Detalles de implementación del algoritmo}
    \subsection{Optimizaciones y mejoras posibles}
    \subsection{Evaluación experimental de rendimiento}
    
\section{Aplicaciones y casos de uso}
    \subsection{Escenarios en los que el algoritmo es particularmente útil}
    \subsection{Ejemplos de aplicaciones prácticas}
    
\section{Avances y variantes del algoritmo}
    \subsection{Desarrollos posteriores al algoritmo original}
    \subsection{Variantes y mejoras propuestas por otros investigadores}
    
\section{Limitaciones y consideraciones adicionales}
    \subsection{Tamaño de los números en los que el algoritmo es efectivo}
    \subsection{Casos en los que otros enfoques pueden ser más adecuados}
    \subsection{Desafíos y obstáculos para su implementación y uso}
    
\section{Conclusiones y perspectivas futuras}
    \subsection{Resumen de los hallazgos clave}
    \subsection{Reflexiones sobre las fortalezas y debilidades del algoritmo}
    \subsection{Posibles direcciones para futuras investigaciones relacionadas}

\newpage

\end{document}

