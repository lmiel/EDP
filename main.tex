\documentclass{article}
\usepackage{lipsum}
\usepackage[spanish]{babel} % Paquete para configurar el idioma en español
\usepackage{titletoc} % Paquete para personalizar la generación de la tabla de contenido
\usepackage[a4paper, margin=4cm]{geometry} % Paquete para ajustar los márgenes de página
\usepackage{parskip} % Paquete para quitar la indentación al inicio de cada párrafo

\setlength{\parskip}{12pt} % Ajusta el valor según tus preferencias


\title {\huge \textbf{El teorema espectral en EDP lineales}}
\author{Lucía Mielgo}
\date{\today}
\begin{document}
\maketitle % Genera la portada
\newpage

\renewcommand{\contentsname}{Índice} 
\tableofcontents
\newpage
\section{Introducción}
Este trabajo constituye una breve introducción al Teorema espectral aplicado a las ecuaciones en derivadas parciales lineares.

Las ecuaciones en derivadas parciales (EDP) se pueden considerar modelos de evolución en la ciencia y la tecnología. Las EDP describen cómo cambia una determinada cantidad o variable a lo largo del tiempo. Esta cantidad o variable puede representar una amplia variedad de objetos y fenómenos, utilizandose para representar la propagación de ondas, la transferencia de calor, modelar el flujo de fluidos y muchas otras aplicaciones. En pocas palabras, las EDP capturan la idea fundamental de cómo los sistemas se transforman y se desarrollan a lo largo del tiempo.

El teorema espectral en su forma general para operadores autoadjuntos fue establecido por David Hilbert en la primera década del siglo XX. Hilbert demostró que todo operador autoadjunto en un espacio de Hilbert tiene una base ortogonal de autovectores asociados a autovalores reales.
Este teorema nos puede resultar muy util para el álgebra lineal y el analisis funcional. En el contexto de las EDP, el teorema espectral proporciona información importante sobre las soluciones y propiedades de estas ecuaciones.

El objetivo principal de este trabajo de profundización es explorar y analizar la aplicación del teorema espectral en el estudio de las ecuaciones en derivadas parciales lineales. Investigando cómo el teorema espectral puede utilizarse para comprender mejor las propiedades de las soluciones de EDP, así como para abordar problemas de autovalores asociados a operadores lineales involucrados en las ecuaciones.

\newpage

\section{Fundamentos teóricos}
    \subsection{Breve introducción a las EDP}
    Primero, como concepto necesario a entender el teorema espectral, se requiere estar familiarizado con las ecuaciones en derivadas parciales más importantes y que más se utilizan en el teorema espectral.

    Una Ecuacion en Derivadas Parciales es una relacion de la forma:
    
    \begin{equation}
        \quad F\left(x, t, u, \frac{\partial u}{\partial x}, \frac{\partial u}{\partial t}, \ldots, \frac{\partial^2 u}{\partial x^2}, \frac{\partial^2 u}{\partial t^2}, \ldots\right) = 0
        \end{equation}
        
        En esta ecuación, se representa una EDP de primer orden y en una dimensión, que es una de las formas más simples de una EDP. La ecuación representa una relación entre una función desconocida $u(x,t)$ y sus derivadas parciales con respecto a las variables $x$ y $t$
            
        \begin{itemize}
            \item El término $\frac{\partial u}{\partial t}$ representa la derivada parcial de la función $u$ con respecto a la variable temporal $t$. Indica el cambio de $u$ con respecto al tiempo.
            
            \item El término $c\frac{\partial u}{\partial x}$ representa la derivada parcial de la función $u$ con respecto a la variable espacial $x$. Describiendo el cambio de $u$ con respecto a la posición en la dirección $x$. Se puede interpretar como la propagación de la función $u$ en esa dirección.

        \end{itemize}

 
        En este trabajo se van considerar las EDP lineales de segundo orden, que son las más sencillas de resolver y las que más se utilizan en el teorema espectral. Se pueden expresarse de forma general como:

        \begin{equation}
        A\frac{\partial u}{\partial x\partial t}+B\frac{\partial^2 u}{\partial x^2} +C\frac{\partial^2 u}{\partial t^2} +D= 0
        \end{equation}
        
        Donde $u(x,t)$ es la función desconocida dependiente de las variables $x$ y $t$, y donde A,B,C son funciones de $x$ y de $t$. D es una funcion de $x,y,u$ $\frac{\partial u}{\partial x}$ y $\frac{\partial u}{\partial t}$

        Dependiendo de los valores de los coeficientes de los términos de la segunda derivada A,
        B y C, la anterior ecuación puede clasificarse en tres categorías diferentes. Cuando B\textsuperscript{2} - 4AC $<$ 0 es una ecuación parabólica, si Si B\textsuperscript{2} - 4AC $=$ 0 es una ecuación hiperbólica y si no, B\textsuperscript{2} - 4AC $>$ 0 se considera una ecuación elíptica. También conocidas como ecuación del calor, ecuación de onda y ecuación de Laplace respectivamente. Estas dos primeras, se utilizarán mas adelante en el trabajo ya que describen la mayoría de los problemas físicos y de ingeniería de importancia práctica. Tienen la siguiente forma por orden:
        \begin{equation}
        \frac{\partial u}{\partial t} = \alpha \frac{\partial^2 u}{\partial x^2}
        \end{equation}
        \begin{equation}
        \frac{\partial^2 u}{\partial t^2} = c^2 \frac{\partial^2 u}{\partial x^2}
        \end{equation}
        \begin{equation}
            \frac{\partial^2 u}{\partial t^2} + \frac{\partial^2 u}{\partial x^2}= 0
        \end{equation}
    \subsection{Conceptos básicos del teorema espectral y su aplicación en problemas de autovalores}

    A continuación, se presentan los conceptos básicos del teorema espectral y su aplicación en problemas de autovalores en el contexto de las EDP:

    El teorema espectral se aplica a el espacio Hilbert, un espacio vectorial en el que se pueden definir operaciones de suma y multiplicación por escalares. El cual se distingue del resto de espacio vectoriales porque existe un producto interno, que toma dos elementos del espacio y da lugar a una norma. Tal que:
    \begin{equation}
        |ab| = \sqrt{\langle ab, ab \rangle} = \sqrt{\langle a,a \rangle \langle b,b \rangle}
        \end{equation}
        
        Donde $a$ y $b$ son elementos de un espacio de Hilbert, $\langle \cdot, \cdot \rangle$ representa el producto interno y $| \cdot |$ es la norma.
    
    
    También se aplica en la resolución de problemas de autovalores, ya que nos permite determinar los autovalores y autovectores asociados a un operador lineal, quienes nos proporcionan informacion sobre las propiedades del sistema.

    Y para el estudio de las EDP es una herramienta fundamental ya que establece que las soluciones de estas ecuaciones pueden expresarse como combinaciones lineales de los autovectores asociados a un operador autoadjunto. Estos autovectores representan modos o frecuencias características de la ecuación y nos permiten comprender mejor el comportamiento de las soluciones.
    
    En resumen, el teorema espectral nos permite descomponer operadores autoadjuntos en términos de sus autovectores y autovalores, lo que resulta útil para abordar problemas de autovalores en el contexto de las EDP y obtener soluciones especiales que capturan las propiedades del sistema.


\newpage

\section{El teorema espectral en EDP lineales}
    \subsection{Análisis de las EDP lineales y su formulación matemática}
    Ya hemos visto anteriormente una breve introducción a las EDP. Asi que ahora vamos a ver su formulación matemática y que representan. 

    Las ecuaciones parabólicas, hiperbólicas y elípticas, es decir, la ecuación del calor, la ecuación de ondas y la ecuación de Laplace respectivamente, son los modelos más clásicos y representativos en el contexto de las ecuaciones en derivadas parciales de evolución. Estas ecuaciones tienen características matemáticas muy diferentes. 

    La ecuación del calor describe fenómenos altamente irreversibles en el tiempo, donde la información se propaga a una velocidad infinita. Se puede representar de la siguiente forma:
    \begin{equation}
        \frac{\partial u}{\partial t} = \alpha \frac{\partial^2 u}{\partial x^2}
        \end{equation}
    
        Donde $u(x,t)$ es la temperatura en un punto $x$ en el tiempo $t$, y $\alpha$ es el coeficiente de difusión térmica.

    Por otro lado, la ecuación de ondas es el prototipo de un modelo de propagación con velocidad finita y completamente reversible en el tiempo.
    \begin{equation}
        \frac{\partial^2 u}{\partial t^2} = c^2 \frac{\partial^2 u}{\partial x^2}
        \end{equation}
        
        Donde $u(x,t)$ es la función de onda que representa la perturbación en un punto $x$ en el tiempo $t$, y $c$ es la velocidad de propagación de la onda.

    La ecuación de Laplace describe el equilibrio térmico o el estado estacionario en un sistema. Se puede representar de la siguiente manera:

        \begin{equation}
            \frac{\partial^2 u}{\partial t^2} + \frac{\partial^2 u}{\partial x^2}= 0
        \end{equation}

    También se puede encontrar escrita como:

        \begin{equation}
            \Delta u = 0
            \end{equation}
    Donde $\Delta u$ es el operador de Laplace o "Laplaciano" de la función $u$.

    En resumen, la ecuación del calor captura procesos de difusión y disipación, mientras que la ecuación de ondas representa la propagación de perturbaciones a través de un medio y la ecuación de Laplace representa el equilibrio o estado estacionario de un sistema. En el contexto de las EDP, el teorema espectral proporciona información importante sobre las soluciones y propiedades de estas ecuaciones.

    El análisis de las EDP lineales implica estudiar las propiedades y características de las soluciones de estas ecuaciones, como su existencia, unicidad, regularidad y comportamiento asintótico. Sus soluciones pueden variar dependiendo de las condiciones iniciales y de contorno, así como de los parámetros involucrados en la ecuación. Por lo tanto, es importante comprender cómo se comportan las soluciones en diferentes situaciones y cómo se ven afectadas por los cambios en los parámetros.

    \subsection{Descripción del teorema espectral en el contexto de EDP lineales}

    En el teorema espectral, se busca descomponer un operador autoadjunto en términos de sus autovectores y autovalores asociados. Un autovector es un vector no nulo que, al aplicar el operador, solo se escala por un factor constante, representado por el autovalor correspondiente.
    Sea $A$ un operador lineal y autoadjunto en un espacio de Hilbert $\mathcal{H}$. Un autovector $v$ de $A$ es un vector no nulo tal que:

    \begin{equation}
    Av = \lambda v
    \end{equation}
    
    Donde $\lambda$ es el autovalor correspondiente al autovector $v$. El autovalor $\lambda$ es un número escalar que representa la escala por la cual el autovector es transformado al aplicar el operador $A$.
    
    Entonces, para un operador autoadjunto, los autovectores son los vectores que se mantienen en la misma dirección o una paralela al aplicar el operador, y los autovalores son los factores de escala asociados a esos autovectores.
    
    La descomposición espectral de un operador autoadjunto significa que podemos expresar el operador como una combinación de sus autovectores y autovalores correspondientes. Esto nos permite descomponer el operador en términos de las propiedades especiales de estos vectores y valores, lo que a su vez nos proporciona información importante sobre las características y el comportamiento del operador.. Esto se puede representar de la siguiente manera:
    
    \begin{equation}
    A = \sum_{i} \lambda_i \langle v_i, \cdot \rangle v_i
    \end{equation}
    
    Donde $\lambda_i$ son los autovalores, $v_i$ son los autovectores y $\langle v_i, \cdot \rangle$ representa el producto interno entre el autovector $v_i$ y un vector del espacio.
    
    Esta descomposición espectral nos permite entender mejor las propiedades y el comportamiento del operador $A$ y nos proporciona una forma conveniente de representar soluciones de ecuaciones en derivadas parciales lineales asociadas a este operador.
    
    También establece que los operadores autoadjuntos en espacios de Hilbert pueden ser diagonalizados, es decir, descompuestos en términos de sus autovectores y autovalores. Esta diagonalización permite estudiar y analizar las propiedades del operador de manera más simple y efectiva. Además, el teorema espectral muestra que estos autovectores forman una base ortogonal del espacio de Hilbert.

    Aplicación en problemas de autovalores en EDP: En el contexto de las EDP, el teorema espectral es utilizado para resolver problemas de autovalores asociados a operadores lineales involucrados en las ecuaciones. Estos problemas se presentan al buscar soluciones especiales de las EDP que satisfacen una relación de proporcionalidad con respecto al operador. Los autovalores y los autovectores obtenidos a través del teorema espectral permiten describir las propiedades de estas soluciones y su comportamiento frente a las EDP.

    La aplicación del teorema espectral en problemas de autovalores en EDP proporciona una herramienta poderosa para comprender y resolver estas ecuaciones, ya que permite descomponer los operadores involucrados en términos de sus autovectores y autovalores, lo que simplifica su análisis. Al obtener los autovectores y autovalores correspondientes, se pueden determinar las soluciones especiales de las EDP y estudiar su comportamiento, estabilidad y propiedades fundamentales.

    La aplicación del teorema espectral en problemas de autovalores en EDP tiene diversas implicaciones. Por ejemplo, puede ayudar a clasificar las soluciones de las EDP según sus modos o frecuencias características, lo que permite comprender mejor los patrones de oscilación o propagación presentes en el sistema. Además, la diagonalización de los operadores autoadjuntos permite simplificar los cálculos y la resolución numérica de las EDP, ya que se puede trabajar con una base ortogonal de autovectores.

    \subsection{Ejemplos y aplicaciones del teorema espectral en EDP lineales}
    Ejemplo de resolución de la ecuación de onda utilizando el teorema espectral:

    Consideremos la ecuación de onda unidimensional en un dominio acotado, dada por:
    \begin{equation}
    \frac{{\partial^2 u}}{{\partial t^2}} = c^2 \frac{{\partial^2 u}}{{\partial x^2}}
    \end{equation}

    Donde \(u(x, t)\) es la función desconocida que representa la onda en el dominio, \(c\) es la velocidad de propagación de la onda y \(x\) y \(t\) son las variables espacial y temporal, respectivamente.
    
    Para resolver esta ecuación utilizando el teorema espectral, primero debemos encontrar los autovectores y autovalores del operador diferencial \(\frac{{\partial^2}}{{\partial x^2}}\). Estos autovectores son funciones especiales que cumplen con ciertas propiedades y representan los modos normales de vibración de la onda.
    
    Supongamos que encontramos los autovectores \(\phi_n(x)\) y los autovalores \(\lambda_n\) asociados al operador diferencial. Luego, la solución de la ecuación de onda puede expresarse como una combinación lineal de estos autovectores:
    
    \begin{equation}
    u(x, t) = \sum_{n=1}^{\infty} A_n \phi_n(x) \cos(\omega_n t + \phi_n)
    \end{equation}
    
    Donde \(A_n\) son coeficientes que determinan la amplitud de cada modo normal, \(\omega_n\) es la frecuencia angular asociada al modo y \(\phi_n\) es una fase específica.
    
    La descomposición espectral nos permite entender cómo cada modo normal contribuye a la forma y evolución de la onda en el tiempo. Al determinar los coeficientes \(A_n\) y las frecuencias \(\omega_n\) correspondientes, podemos obtener la solución completa de la ecuación de onda y analizar su comportamiento dinámico. En situaciones más complejas, pueden considerarse condiciones iniciales y de contorno específicas, lo que afectaría la elección de los autovectores y los coeficientes asociados.
    Consideremos la ecuación del calor en una barra unidimensional, dada por:

    \begin{equation}
    \frac{{\partial u}}{{\partial t}} = k \frac{{\partial^2 u}}{{\partial x^2}}
    \end{equation}

    Donde \(u(x, t)\) es la temperatura en el punto \(x\) y tiempo \(t\), y \(k\) es el coeficiente de difusión térmica.

    Para aplicar el teorema espectral, primero debemos encontrar los autovectores y autovalores del operador diferencial \(\frac{{\partial^2}}{{\partial x^2}}\). Estos autovectores, que son soluciones de la ecuación de autovalores, representan los modos de temperatura estacionarios de la barra.

    Supongamos que encontramos los autovectores \(\phi_n(x)\) y los autovalores \(\lambda_n\) asociados al operador diferencial. Luego, la solución de la ecuación del calor puede expresarse como una combinación lineal de estos autovectores, ponderados por coeficientes \(A_n\) que determinan la contribución de cada modo:

    \begin{equation}
    u(x, t) = \sum_{n=1}^{\infty} A_n \phi_n(x) e^{-\lambda_n k t}
    \end{equation}

    Donde \(e^{-\lambda_n k t}\) es el factor de atenuación que describe la evolución temporal de cada modo.

    La descomposición espectral nos permite comprender cómo cada modo de temperatura afecta la distribución de calor en la barra a lo largo del tiempo. Al determinar los coeficientes \(A_n\) correspondientes a las condiciones iniciales, podemos obtener la solución completa de la ecuación del calor y analizar cómo se propaga y se difunde el calor en el sistema.

    Este ejemplo básico muestra cómo el teorema espectral se aplica en la resolución de la ecuación del calor. En casos más complejos, se pueden considerar condiciones de contorno y fuentes de calor adicionales, lo que afectaría la elección de los autovectores y los coeficientes asociados.


    El teorema espectral también es útil para abordar problemas de autovalores asociados a operadores lineales involucrados en las ecuaciones. Estos problemas se presentan con frecuencia en el estudio de las EDP, ya que permiten determinar las propiedades fundamentales de los sistemas.

    Supongamos que tenemos un operador lineal autoadjunto \(L\) que actúa en un espacio de funciones adecuado. El teorema espectral establece que este operador se puede descomponer en términos de sus autovectores y autovalores asociados.

    Los autovectores del operador \(L\), denotados como \(\phi_n\), son funciones no nulas que satisfacen la ecuación \(L\phi_n = \lambda_n \phi_n\), donde \(\lambda_n\) son los autovalores correspondientes.Y como forman una base ortogonal, cualquier funcion del espacio se puede expresar como una combinación lineal de los autovestores.
    \begin{equation}
    f(x) = \sum_{n} c_n \phi_n(x)
    \end{equation}
    Donde \(c_n\) son los coeficientes de la descomposición.

    Y con esta descomposicion espectral podemos entender mejor las soluciones de las EDP y utilizarla para encontrar autovalores y autovestores asociados al operador \(L\).

\newpage

\section{Conclusiones}
\newpage

\section{Referencias bibliográficas}

\renewcommand{\refname}{}
\begin{thebibliography}{9}

    \bibitem{nice}
    Desconocido. \emph{Partial differential equations and spectral theory, Université de Nice 2018-2019}.
    
    \bibitem{martin}
    Gómez, J. D. M. y Anguas, J. R. \emph{Ecuaciones en Derivadas Parciales y análisis funcional}.
    
    \bibitem{laugesen}
    Laugesen, R. S. \emph{Spectral Theory of Partial Differential Equations}. University of Illinois at Urbana-Champaign.
    
    \bibitem{zuazua}
    Zuazua, E. \emph{Ecuaciones en Derivadas Parciales}.
    
    \end{thebibliography}

\end{document}

